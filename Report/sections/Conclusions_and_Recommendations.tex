\section{Conclusions and Recommendations}
The question to be answered was what cashier schedule for the fuel station would maximize the profit, while meeting the following requirements:
\begin{enumerate}
	\item Cars (with  trailers) typically spend at most 30 minutes from arrival to departure.
	\item Trucks typically spend at most 45 minutes from arrival to departure.
	\item At most 1$\%$ of the arriving vehicles will be blocked.
\end{enumerate}
Furthermore, Bill asked to find a schedule for when the demands would decrease (due to cars becoming more fuel efficient).

In section \ref{sec:simulationresults} we described the results of the simulations done with OptQuest. The optimal solution has 21 cashiers working a shift, which costs \EUR{2242,50} a day, which is a 30$\%$ improvement with respect to a schedule where 5 cashiers would be working at all times.

With this optimal solution, the requirements are still fulfilled:
\begin{enumerate}
	\item Cars spent about 25 minutes on average from arrival to departure.
	\item Trucks spent about 29 minutes on average from arrival to departure.
	\item 0.53$\%$ of all vehicles are blocked, which is below the threshold of 1$\%$.
\end{enumerate}

Regarding the schedule for decreased demands, we tested the required schedules for 20$\%$ and 30$\%$ less arrivals. With 20$\%$ less arrivals we required a schedule with 68.63 hours which cost \EUR{1715,75}. This is a 46$\%$ improvement over the naive schedule mentioned above. With 30$\%$ less arrivals we needed 59.92 hours for a total cost of \EUR{1498} per day. This is a 53$\%$ improvement over the naive schedule. 

With these results, the block rate would not be violated (0.7$\%$ and 0.8$\%$ for -20$\%$ and -30$\%$ respectively). But the throughput times were  violated for cars, as they spent about 40 minutes and about 35 minutes on average from arrival to departure, for -20$\%$ and -30$\%$ arrival rate respectively. So the schedules found by OptQuest did not completely satisfy the constraints. We could not find the cause of this problem, but as the solutions satisfy most constraints (only the throughput time for cars was violated and not by much), we still consider the schedules to be fairly reliable.

So all in all we found a good solution which saves Bill 30$\%$ on cashier costs for the current demand, and fairly good solutions for decreased demands (as one constraint was violated a little) where the cashier cost was decreased by 46-53$\%$.

Regarding recommendations, we would suggest to enlarge the checkout queue so vehicles will not block the pump when they are done refueling and are waiting for the cashier to become available.
Furthermore it would be an idea to pay unattented at a barrier so people can pay with a card. This would reduce the cashier cost and would allow more people to pay at the same time, reducing the time the vehicles have to wait for payment.
Lastly, the selection of the lane could be guided by electronic signs above the lanes with the expected waiting time, as the shortest queue not always has the shortest waiting time.