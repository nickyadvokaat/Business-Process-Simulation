\section{Experimental Setup}
In our experiment, we will be using the OptQuest tool that Arena offers natively to find an optimal solution.
Our model contains several decision variables that can be altered to find the optimal solution.
OptQuest allows us to set bounds to decision variables, as well as impose constraints on them and on outcomes.
This is used to reduce the size of the solution space and allow OptQuest to automatically find the minimum cost.

The only decision variables are the cashier schedules.
There are 12 shift resources, which define the number of cashiers in each 4-hour shift.
Since there are only 5 cash registers, the number of cashiers in a shift can not exceed 5.
As such, each shift variable is be bounded between 0 and 5.
However, this leads to a rather large solution space, as there are $6^{12}$ possibilities. 

This number is reduced by several constraints.
First, the percentage of vehicles blocked should not exceed 1 percent of the total number of vehicles arrived. 
Since there are 12 shifts of 4 hours, every shift overlaps with two other shifts.
As such, it might occur that there are more cashiers active than there are cash registers.
To combat this problem, and reduce the size of the solution space, we state that the number of workers active at any time should be less than or equal to 5, the number of cash registers. 
We can then calculate the cost of the given schedule by adding the number of cashiers in every shift, multiplying it by 4, the number of hours in a shift, and multiplying it again by 25 Euro, the wage of a cashier per hour.
Our goal is then to minimize this cost variable. 

We attempt to find a near optimal solution in several steps.
The first step is to run 10000 simulations with only 1 replication.
Next, we increase the number of replications to 7 (one week) and run with only 300 simulations, because in the first step, the optimal solution was found between simulation 250 and 300.
Since the way OptQuest iterates over the data does not guarantee an optimal solution, the best solution found by OptQuest might not be the optimal solution.
As such, instead of simply taking the best solution found by Optquest, we will take several of the best solutions found, and try to optimize these by hand, by removing cashiers from several shifts, whilst ensuring that the requirements are still satisfied. 
