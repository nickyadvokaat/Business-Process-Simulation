\section{Simulation Model Description}

In this section the Arena model is described and explained. The model has been split into two models, firstly the management of the cashiers (main model) and secondly the management of the vehicles (submodel Pumps). The main model can be found in figure \ref{fig:main}. Larger models can be found in appendix \ref{app:modeldescription}. The main model and sub model will be described briefly in this section. The detailed explanations can be also be found in appendix \ref{app:modeldescription}.  The used variables, attributes and schedules will be described as well.
\begin{figure}[h!]
\begin{center}
	\includegraphics[scale=0.6]{images/model-description/main}
	\caption{Arena Main Model}
	\label{fig:main}
\end{center}
\end{figure}
\subsection{Main Model Cashier management}
The main model is the model which describes the cashier management. What this model does is that it creates five checkouts at the start. These checkouts are the entities which go through the system. When cashiers arrive they are put in a set of available cashiers and the checkouts seize a cashier. When a cashier is seized the checkout opens its associated lanes for 4 hours and after that it is decided if another cashier takes over the checkout or if the checkout closes. If it closes all the vehicles currently in the lane will be served and and then the cashier leaves. If another cashier takes over the checkout the old cashier restocks the supplies and goes home. Just before the cashier finishes its shift a duplicate of the checkout is created, since the old entity of this checkout leaves the system after the shift. The duplicate behaves the same as the old entity and the model continues with still five checkouts. 

\subsection{Vehicle management}
The other (sub)model in is the Vehicle management model. This model models the arrival of vehicles at the fuel station and how these vehicles pass through the model. A detailed figure of the model can be found in figure \ref{fig:modelpumps} in appendix \ref{app:modeldescription}. In this sub model for every vehicle an entity is created. This entity first checks if one of the open lanes has enough space so the vehicle can enter the queue for the lane. If multiple of these lanes have enough space the entity chooses the shortest lane. After that it waits till its turn and refuels and after that it joins the queue for the cashier. This queue is special , because in that queue three lanes merge into one. When the vehicle paid the cashier the vehicle leaves the model and the queue is updated. 

\subsection{Resource schedules, variables and attributes}
All the resource schedules, variables and attributes which have been used in these models will be described in appendix  \ref{app:modeldescription}.