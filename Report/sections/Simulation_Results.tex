\section{Simulation Results}
In this section we will look at the results of the simulatios done with OptQuest as described in section 7. 
The simulations consisted of four different rounds. 
In the first round we did about 10,000 simulations of 1 replication. 
This took quite some time to finish (about a day).
As we noted between these results we saw that the best solution was a simulation between the 200 and 300.
So after that we did again about 300 solutions with 7 replications instead of 1.
This gave us only a little amount of feasible solutions. 
These solutions were again simulated only now with 30 replications for each simulation.
This gave us a solution which was not bad at all, but it was not optimal as we have shown in section 8.
For the last part we tried to adapt the solution OptQuest gave us manually.
This lead to a better solution, because we could decrease the number of cashiers for Shift 1, with 1.
The solution we eventually have is shown in \autoref{tab:opt}.

\begin{table}
	\centering
	\caption{optimal solution}
	\begin{tabular}{l | l}
		Shift number & Number of cashiers in this shift\\
		A\_S1 & 1 \\
		A\_S2 & 2 \\
		A\_S3 & 1 \\
		A\_S4 & 3 \\
		A\_S5 & 1 \\
		A\_S6 & 2 \\
		A\_S7 & 1 \\
		A\_S8 & 3 \\
		A\_S9 & 1 \\
		A\_S10 & 3 \\
		A\_S11 & 0 \\
		A\_S12 & 3 \\
	\end{tabular}
	\label{tab:opt}
\end{table}

\subsection{Optimal Solution}
The optimal cashier schedule for Bill's fuel station is shown in figure \ref{fig:schedule}.
This schedule has 21 rows and every row represents a cashier working a shift.
This means for each day there are 21 cashiers scheduled, which is much better than the 30 cashiers in a naive schedule.

\newcommand{\cc}{\cellcolor{cyan}}

\begin{figure}
	\centering

	\renewcommand{\arraystretch}{0.6} % this reduces the vertical spacing between rows
	
\begin{tabular}{|c|c|c|c|c|c|c|c|c|c|c|c|}
	\hline
	0:00 & 2:00 & 4:00 & 6:00 & 8:00 & 10:00 & 12:00 & 14:00 & 16:00 & 18:00 & 20:00 & 22:00  \\
	\hline
	\hline
	\cc & \cc & & & & & & & & & & \\ \hline 
	& \cc & \cc & & & & & & & & & \\ \hline
	& \cc & \cc & & & & & & & & & \\ \hline
	& & \cc & \cc & & & & & & & & \\ \hline
	& & & \cc & \cc & & & & & & & \\ \hline
	& & & \cc & \cc & & & & & & & \\ \hline
	& & & \cc & \cc & & & & & & & \\ \hline
	& & & & \cc & \cc & & & & & & \\ \hline
	& & & & & \cc & \cc & & & & & \\ \hline
	& & & & & \cc & \cc & & & & & \\ \hline
	& & & & & & \cc & \cc & & & & \\ \hline
	& & & & & & & \cc & \cc & & & \\ \hline
	& & & & & & & \cc & \cc & & & \\ \hline
	& & & & & & & \cc & \cc & & & \\ \hline
	& & & & & & & & \cc & \cc & & \\ \hline
	& & & & & & & & & \cc & \cc & \\ \hline
	& & & & & & & & & \cc & \cc & \\ \hline
	& & & & & & & & & \cc & \cc & \\ \hline
	\cc & & & & & & & & & & & \cc \\ \hline
	\cc & & & & & & & & & & & \cc \\ \hline
	\cc & & & & & & & & & & & \cc \\ 
	\hline
	\hline
	4 & 3 & 3 & 4 & 4 & 3 & 3 & 4 & 4 & 4 & 3 & 3\\
	\hline
\end{tabular}
\caption{The schedule. Each row represents one of 21 shifts.}
\label{fig:schedule}
\end{figure}

The total cost of the solution is \EUR{2242,50} euro, this is \EUR{2100} euro for cashiers working their shift and \EUR{142,50} euro for cashiers working overtime (restocking or continuing their shift until the lanes in front of their checkout are empty). 
This is a lot less than the \EUR{3000} a day the naive schedule would have costed.
This is all shown in table \autoref{tab:costs}.

\begin{table}[h!]
	\centering
\begin{tabular}{|c|c|c|c|}
	\hline
	& Hours & Percentage & Cost\\ \hline \hline
	Scheduled Hours & 84.0 & 93.6\% & \EUR{2100} \\
	Extra Hours & 5.7 & 6.4\% & \EUR{142,50}\\
	\hline
	\hline
	Total & 89.7 & 100\% & \EUR{2242,50}\\
	\hline
\end{tabular}
\caption{Average number of hours worked on a single day}
\label{tab:costs}
\end{table}

There were also other demands of Bill regarding customer satisfaction. 
The first was that the total number of vehicles that are blocked and thus cannot enter the fuel station, can not be higher than 1\%.
As you can see in table \autoref{tab:block} 0.53 \% of all vehicles get blocked. 
Notable is that for cars this is almost nothing (not even 0.1\%) but for trucks it is larger, as many as 3\%.
This can be explained very easily.
When it is very busy at the fuel station all the lanes are full, or nearly full. 
A truck needs a lot of space, while a car does not, and since the arrival rate of cars is much higher than of trucks it is likely that there will not be space for trucks when it is very busy.\\
The last thing was that cars and trucks cannot be too long in the fuel station. 
Cars have to be on the road again in 30 minutes on avarage and trucks in 45 minutes. As you can see in table \autoref{tab:throughput} this is indeed the case.

\begin{table}
	\begin{minipage}{.5\linewidth}
	\centering
	\begin{tabular}{|c|c|}
		\hline
		All Vehicles & 0.53\% \\ \hline
		Cars & 0.034 \% \\ \hline
		Trucks & 3.36\% \\ \hline
	\end{tabular}
	\caption{Block rate percentage}
	\label{tab:block}
	\end{minipage}%
	\begin{minipage}{.5\linewidth}
	\centering
	\begin{tabular}{|c|c|}
		\hline
		\multicolumn{2}{|c|}{Throughput times} \\ \hline
		Cars & 25m:31s \\ \hline
		Trucks & 29m:16s \\ \hline
	\end{tabular}
	\caption{Average throughput time}
	\label{tab:throughput}
	\end{minipage}
\end{table}

\subsection{Optimal Solution for Decreased Demands}
Another question asked by Bill was how must he adapt the schedule such that in the future, when cars become more fuel efficient and the demand goes down, he still has a good schedule. 
To simulate this we lowered the arrival amounts with 10, 20 and 30\% and  also the amount of fuel needid with the same amounts. 
The way of getting this solution was different from from the way we got it for the current schedule.
This is all explained in appendix \autoref{app:demands}.