\section{System Description}
In this section we give a description of the fuel station which is about to be modelled. This means we will describe all the key object classes which appear in the model and all of its associated actions. When these are described the next thing we treat are the performance indicators. These indicators help us to show how good the performances are after a simulation. The last two things treated in this section are the questions which need to be answered with the model and alternative solutions to reach the goal Bills is eager to achieve. 
\\
\\Consists of:\\
- Key Object Classes and Actions\\
- Performance Indicators\\
- Questions to be answered\\
- Solution Alternatives \\
\\
\subsection{Key Object Classes}
In our fuel station we can distinguish several key object classes. All of these classes are stated in table \ref{tab:koc}. The classes we can distinguish are $Vehicle$, $Lanes$, $Employees$, $Pumps$ and $Queue Spot$. The class $Vehicle$ consists of all the types of vehicles that can enter the fuel station. These are all stated in the types, namely $Car, Car \ with \ Trailer$ and $Truck$. All these vehicles have the same states while they are in the fuel station. The firs state is $Arriving$, when a vehicle is in the state $Arriving$ it means that the vehicle has entered the fuel station but has not yet chosen any queue to enter. Obviously the next state $In \ Queue$ means that the vehicle has chosen a queue and is waiting in line. The state $Refueling$ means the vehicle has reached the pump and is now refueling. The state $Waiting \ for \ Cashier$ means the vehicle is done refueling, but has not reached the cashier yet. The next state $Paying$ obviously means the vehicle has reached the cashier and is paying for it's fuel. The last state is the state the vehicle recedes in when it leaves the station or has not entered the station yet, this is the $Idle$ state. \\
\indent The second class is the $Lanes$ class. This class consist of a single type, the $Lane$ type. These $Lanes$ are the queues in front of the pumps. Such a lane can obviously be Available, which means vehicles can enter the queue, closed, vehicles cannot enter the queue, or full, no more new vehicles can enter the queue because the queue is full. Hence the states $Available, Closed$ and $Full$. For these lanes we assume that lanes can only be opened in blocks of three. So no cashier is serving only one or two lanes.\\
\indent The third class is the $Employees$ class. There is only one kind of employee on the fuel station, which is the $Cashier$ type. A cashier can be in the state $Off-duty$, which means the cashier is not working at the moment. When he is in the state $Available$ the cashier is on duty but he is not serving customers at the moment. When he is busy serving a customer he is in the state $Serving$. The last state a cashier can recedes in is the state $Restocking \ Supplies$. This means the shift of the cashier has ended but he is busy restocking the supplies before he goes home. \\
\indent The next class contains the different kind of pumps the fuel station has, hence the name $Pumps$. Each lane has two pumps. A $Low \  Speed \ Pump$ and a $High \ Speed \ Pump$. The first pump is for cars and the second is for trucks. The fuel station has plenty of room underground, therefore every lane can contain one of these pumps. Each of these pumps can be either in the $Available$ or $In \ Use$ state, which means the pump is in use for refueling. It is important to notice that at most one of the pumps in a lane can be in the state $In \ Use$, but both can be in the state $Available$.\\
\indent The last class contains the class $Queue Spot$. This class represents the spots which are available between the pumps and the cashier. There are three kinds of spots. $First \ Spot$ is the spot right after the pump and there are three of them for every set of lanes a cashier is serving. $Second \ Spot$ is the spot after $First \ Spot$ and there are only two of them for every set of lanes. $Third \ Spot$ is the last spot and also the spot where a vehicle is served by a cashier. There is only of them for every set of lanes. Every one of these spots can either be occupied by a vehicle or be free, hence the states $Occupied$ and $Free$.

\begin{center}
\begin{table}[h]
\begin{tabular}{| l | l | l | l | l |}
\hline
\textbf{Vehicles} & \textbf{Lanes} & \textbf{Employees} & \textbf{Pumps} & \textbf{Queue Spot}\\
\hline
\textbf{Types} & \textbf{Types} & \textbf{Types} & \textbf{Types} & \textbf{Types}\\
- Car & - Lane & - Cashier& - Low Speed Pump & - First Spot\\
- Car with Trailer & & & - High Speed Pump & - Second Spot\\
- Truck & & & & - Third Spot\\
& & & & \\
\textbf{States} & \textbf{States} & \textbf{States} & \textbf{States} & \textbf{States}\\
- Arriving & - Available & - Off-duty & - Available & - Free\\
- In Queue & - Closed & - Available & - In Use & - Occupied\\
- Refueling & - Full & - Serving & &\\
- Waiting for Cashier & & - Restocking Supplies & &\\
- Paying & & & &\\
- Idle & & & &\\
\hline
\end{tabular}
\caption{Key Object Classes}
\label{tab:koc}
\end{table}
\end{center}

\subsection{Actions}
There are a number of actions that can be done in this system. Each of these actions causes a number of state changes, depending on the state the system is in when the action occurs. All possible actions together with their state changes, are described below.
\begin{itemize}
\item \textbf{A vehicle arrives}\\
\textbf{State changes:}
\begin{enumerate}
\item Vehicle state changes from \textit{Idle} to \textit{Arriving}.
\item If every lanes is \textit{Closed} or \textit{Full}, then Vehicle becomes \textit{Idle} again.
\item If a lane is \textit{Available} and the Pump is \textit{Available} (No Vehicle using or blocking it) then:
\begin{enumerate}
\item The Pump goes from \textit{Available} to \textit{In Use}.
\item The Vehicle goes from \textit{Arriving} to \textit{Refueling}
\end{enumerate}
Assumption: If a Pump is \textit{Available}, the Lane is empty (i.e. when a Vehicle stops blocking it, the next Vehicle starts using it at the same time).\\
\item If a lane is \textit{Available} and there is enough space for the arriving Vehicle and the Pump is \textit{In Use} then:
\begin{enumerate}
\item the Vehicle goes from \textit{Arriving} to \textit{In Queue}
\item if the Lane is full now, it goes from \textit{Available} to \textit{Full}
%Is deze State
\end{enumerate}
\end{enumerate}
\item \textbf{A cashier starts a shift}\\
\textbf{State changes:}
\begin{enumerate}
\item If there is no Vehicle at the Third Queue Spot of the Cashier, the Employee goes from \textit{Off-duty} to \textit{Available}.
\item If there is a Vehicle at the Third Queue Spot of the Cashier, the Employee goes from \textit{Off-duty} to \textit{Serving}.
\end{enumerate}
\item \textbf{A cashier's regular shift ends}\\
\textbf{State changes:}
\begin{enumerate}
\item If another cashier takes over, the Employee whose shift ends goes from \textit{Available} or \textit{Serving} to \textit{Restocking Supplies}
\item If no cashier takes over, the Lanes of the cashier go from \textit{Available} to \textit{Closed}
\end{enumerate}
\item \textbf{A cashier finishes Restocking Supplies}\\
\textbf{State changes:}
The Employee goes from \textit{Restocking Supplies} to \textit{Off-duty}
\item \textbf{A vehicle finishes refueling}\\
\textbf{State changes:}
\begin{enumerate}
\item If the Vehicle can move completely to the Queue Spots of the Cashier, the Pump goes from \textit{In Use} to \textit{Available}
\item If the cashier of the Lane the Vehicle is in is \textit{Available} then:
\begin{enumerate}
\item The Vehicle goes from \textit{Refueling} to \textit{Paying}
\item The Employee of the Lane the Vehicle is in goes from \textit{Available} to \textit{Serving}
\end{enumerate}
\item If the cashier of the Lane the Vehicle is in is \textit{Serving}, then the Vehicle goes from \textit{Refueling} to \textit{Waiting for Cashier}.
%If lane can be blocked, check if it can be unblocked here
\end{enumerate}
\item \textbf{A vehicle has payed}\\
\textbf{State changes:}
%If the Lanes are closed and empty, if < 10 minute past end of shift cashier goes to Restocking supplies, if > 10 minute past end of shift to Off-duty
\item \textbf{A cashier restocks}\\
\textbf{State changes:}
\item \textbf{A cashier opens lanes}\\
\textbf{State changes:}
\item \textbf{A cashier closes lanes}\\
\textbf{State changes:}
\item \textbf{A cashier starts after shift work}\\
\textbf{State changes:}
\item \textbf{A cashier ends after shift work}\\
\textbf{State changes:}
\end{itemize}
\subsection{Performance Indicators}
\begin{itemize}
\item (Average) throughput times\\
\item Average resource utilization\\
\item Minimum / average / maximum queue sizes\\
\item Cashier costs\\
\item Average waiting time before pump\\
\item Average waiting time after pump
\end{itemize}