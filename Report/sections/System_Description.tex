\section{System Description}
Consists of:\\
- Key Object Classes and Actions\\
- Performance Indicators\\
- Questions to be answered\\
- Solution Alternatives \\
\\
\textbf{Key Object Classes:}\\
In our fuel station we can distinguish several key object clases. All of these classes are stated in table \ref{tab:koc}

\begin{center}
\begin{table}[h]
\begin{tabular}{| l | l | l | l | l |}
\hline
\textbf{Vehicles} & \textbf{Lanes} & \textbf{Employees} & \textbf{Pumps} & \textbf{Queue Spot}\\
- Car & - Lane & - Cashier& - Low Speed Pump & - First Spot\\
- Cart with Trailer & & & - High Speed Pump & - Second Spot\\
- Truck & & & & - Third Spot\\
& & & & \\
\textbf{States} & \textbf{States} & \textbf{States} & \textbf{States} & \textbf{States}\\
- Arriving & - Open & - Off-duty & - Idle & - Free\\
- In Queue & - Closed & -Available & - In Use & - Occupied\\
- Refuelling & - Full & - Serving & &\\
- Waiting for Cashier & & - Restocking Supplies & &\\
- Paying & & & &\\
- Idle & & & &\\
\hline
\end{tabular}
\caption{Key Object Classes}
\label{tab:koc}
\end{table}
\end{center}

Assumption that only blocks of three lanes are openend and not a subset of a block.

\textbf{Actions:}\\
- A vehicle arives\\
- A vehicle leaves/gets blocked\\
- A cashier starts a shift\\
- A cashier ends a shift\\
- A vehicle refuals\\
- A vehicle pays\\
- A vehicle waits for the cashier\\
- A cashier restocks\\
- A cashier opens lanes\\
- A cashier closes lanes\\
- A cashier starts after shift work\\
- A cashier ends after shift work\\

\textbf{Performance Indicators:}\\
- (Average) througput times\\
- Average resource utilization\\
- Minimum / average / maximum queue sizes\\
- Cashier costs\\
- Average waiting time before pump\\
- Average waiting time after pump
