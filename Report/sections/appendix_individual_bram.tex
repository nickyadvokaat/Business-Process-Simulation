\section{Appendix Individual Assignment 0741516}\label{app:indivbram}
\textbf{Currently, arriving vehicles queue up in the shortest queue. Bill is thinking about a way to guide vehicles to a specific queue. Which queue selection mechanism would you suggest to use for that?}\\
\\
Arriving vehicles line up in the shortest queue because they want to wait as short as possible. But it is not guaranteed that the shortest queue also has the shortest wait time. The wait time depends on how long the vehicles currently in the queue will block the pump. This is also affected by the vehicles in the other lanes with the same cashier, as a vehicle could still block the pump when it cannot fully move into the cashier's queue.\\
What Bill could do is estimate the time it would take for a vehicle to get through the entire process of waiting, refueling and paying for each lane. To estimate that time for a specific lane, he can multiply the number of trucks in that lane by the average refuel time of a truck, do the same for cars and cars with trailers, and then add the results to get an estimated refueling time for the vehicles currently in the queue.\\
Furthermore, he can estimate a time for the pump to free up after a vehicle finishes refueling. This is the time it takes to clean the windows and the average wait time for the necessary cashier queue spots to free up. Of course, this is determined per type of vehicle again. So the estimated time for the pump to free up after refueling is: the average time it takes to clean the windows + the average time it takes for enough spots to free up, dependant on the vehicle type. Estimate this time for each vehicle in the lane and you get the total time for the pump to free up after it has been used.\\\\
Finally, add the total times for refueling and for the pump to free up and you get an estimate for the wait time until the vehicle can start refueling. The lane with the shortest time would be the preferred lane so this lane would be selected for the vehicle.