\section{Appendix Individual Assignment 0747896}\label{app:indivrobbert}
\textbf{As you may know, Luxembourg is a very beautiful country for motorbikes to ride through. However, they seem to be avoiding Bills fuel station. Do you have any suggestions for attracting motorbikes?}

For more detailed analysis, we would need to know arrival counts as provided for cars, cars with trailers and trucks.
The question suggests that a lot of motorbikes pass by but no use the fuel station.
There can be more reasons for this that just the layout/operation of this fuel station. 
For example, it may be that they all know that fuel is cheaper in Luxembourg than in surrounding countries and therefore postpone refuelling until their arrival in the country and tank at the first possible stop.

Motorbikes typically have much smaller fuel tanks than cars and trucks.
This implies that their refuelling time would also be much shorter.
Which could imply that motorbike drivers are less willing to wait a long time before they can refuel. 
Much like in a supermarket, where you do not want to wait for other people with a lot of groceries if you only have a carton of milk whereas if you have a full cart too, you are more willing to wait for others.

The first suggestion is to count the actual number of motorbikes that pass the fuel station but not use it.
Depending on the numbers, several alternatives can be considered.
First, a lane only for motorbikes could be opened.
This would reduce the waiting time for motorbikes in general.
Another consideration could be to prioritize motorbikes over other vehicles.
Much like in a traffic jam, where motorbikes are allowed to overtake by driving between two lanes jammed with cars.
This should only be implemented in a situation with few motorbikes, otherwise the other vehicles would have to wait too long.
Finally, we can consider adding features to make the waiting more comfortable for motorbike drivers e.g. a roof over the entire waiting lane such that they stand dry.


