\section{Simulation Results for Decrease in Demands}\label{app:demands}
Bill also asked how his cashier schedule should be modified in the future, to ensure that when cars become more fuel efficient and have to visit less often his costs are still minimized. 
For this we used the same Arena model as for the normal schedule only with 20 and 30 \% less arrivals and 20 and 30 \% less fuel bought. 
To find optimal solutions for these new instances we did some simulations running OptQuest. 
In this case there were 2 rounds.
In the first round for -20\% we did a 1000 simulations with each 1 replication. 
At the start we used the solution of the normal schedule as the suggested solution, because solutions for this schedule could never be worse than the original schedule.
At the end we chose the best solution and did another 50 replications in Arena to check whether this solution was indeed valid. 
We did the same for -30\%, except here we used the solution for -20\% as suggested solution.
The resulting schedules are shown in table \autoref{tab:optdec}.

\begin{table}[h!]
	\centering
	\begin{tabular}{l | l | l }
		Shift number & \#Cashiers for -20\% & \#Cashiers for -30\% \\
		A\_S1 & 1 & 0\\
		A\_S2 & 1 & 1\\
		A\_S3 & 1 & 1\\
		A\_S4 & 2 & 2\\
		A\_S5 & 1 & 1\\
		A\_S6 & 1 & 1\\
		A\_S7 & 1 & 1\\
		A\_S8 & 2 & 1\\
		A\_S9 & 2 & 2\\
		A\_S10 & 2 & 2\\
		A\_S11 & 1 & 0\\
		A\_S12 & 2 & 2\\
	\end{tabular}
	\caption{optimal solution for decreased amounts}
	\label{tab:optdec}
\end{table}

The total costs of the schedules can be found in table \autoref{tab:costsdec}. As you can see this is alread a lot less than the original schedule, so that is good. The second scenario already has less than half of the costs the naive schedule had.

\begin{table}[h!]
	\centering
	\begin{tabular}{c|c|c|c|c|c|c|}
		\cline{2-7}
		& \multicolumn{3}{c|}{-20\%} & \multicolumn{3}{c|}{-30\%} \\ \cline{2-7} 
		& Hours     & Percentage     & Cost &  Hours     & Percentage      & Cost     \\ \hline
		\multicolumn{1}{|c|}{Scheduled hours} & 64.00     & 93.3\%    & \EUR{1600}     & 56.00     & 93.5\%   & \EUR{1400}      \\ \hline
		\multicolumn{1}{|c|}{Extra hours}  & 4.63      & 6.7\%   & \EUR{167,50}       & 3.92      & 6.5\%      &\EUR{98}    \\ \hline
		\multicolumn{1}{|c|}{Total}       & 68.63     & 100\%   &    \EUR{1767,50}   & 59.92     & 100\%    &    \EUR{1498}  \\ \hline
	\end{tabular}
	\caption{Average number of hours worked on a single day for reduced traffic}
	\label{tab:costsdec}
\end{table}
\newpage
In table \autoref{tab:blockReddec} and table \autoref{tab:throughputDecdec} the blockrates and throughput times of these solutions can be found. As you can see all of them still adhere to the constraints.
\begin{table}
	\centering
	\begin{minipage}{.5\linewidth}
	\begin{tabular}{l|l|l|}
		\cline{2-3}
		 & -20\% & -30\% \\ \hline
		\multicolumn{1}{|l|}{All vehicles}  & 0.7\% & 0.8\% \\ \hline
		\multicolumn{1}{|l|}{Cars}         & 0.2\% & 0.5\% \\ \hline
		\multicolumn{1}{|l|}{Trucks}        & 3.9\% & 2.7\% \\ \hline
	\end{tabular}
	\caption{Block rate percentage}
	\label{tab:blockReddec}
	\end{minipage}%
	\begin{minipage}{.5\linewidth}
	\centering
	\begin{tabular}{c|c|c|}
		\cline{2-3}
		\multicolumn{1}{l|}{}        &  \multicolumn{1}{l|}{-20\%} & \multicolumn{1}{l|}{-30\%} \\ \hline
		\multicolumn{1}{|c|}{Cars}  & 39m:54s                    & 35m:36s                    \\ \hline
		\multicolumn{1}{|c|}{Trucks}  & 40m:22s                    & 36m:19s                    \\ \hline
	\end{tabular}
	\caption{Average throughput}
	\label{tab:throughputDecdec}
	\end{minipage}
\end{table}